\documentclass[a4paper]{letter}
\usepackage{url}

\begin{document}

\begin{letter}{}
\vspace*{-5.0cm}

\opening{Dear editor:}

Enclosed you will find a manuscript entitled: ``A simple null model for inferences from network enrichment analysis''. The manuscript describes an in our minds natural null model for network enrichment analysis, as well as two implementations of the null model as an exact test and a sampling method.

Traditionally, the significance of inferences from NEA has been assigned using parametric models, that are tuned by fitting parameters by sampling random network perturbations.  In practice, such perturbations are hard to conduct in a manner that they still resemble the original network, and hence are difficult to use to calculate accurate statistics. In general, it is easier to understand null models that relate to user actions, than more complex null models that relate to model parameters. We also demonstrate our implementations to render far better calibration than the state-of the-art method, BinoX.

\vspace{0.15cm}

The manuscript is acompanied by an open github acount with all software and all tests that we conducted, available for reviewers, or the general public. The manuscript is an original contribution and none of the work it describes has been published elsewhere except for in abstract form. The manuscript will remain with you and will not be submitted elsewhere, until you made a decision as to its suitability for publication in PLOS ONE.

We suggest the following reviewers:

\begin{itemize}

\item Dr. Lieven Clement,
Gent University, \\
\url{Lieven.Clement@ugent.be}

\item Dr. Andrey Alexeyenko, Karolinska Institutet, \\
\url{Andrey.Alexeyenko@ki.se}

\item Professor Ernst C. Wit, University of Groningen\\
\url{e.c.wit@rug.nl}

\item Professor Alfonso Valencia, Barcelona Supercomputing Center,\\
\url{alfonso.valencia@bsc.es}

\item Professor Reinhard Schneider,  Luxembourg Centre for Systems Biomedicine \\
\url{Reinhard.Schneider@uni.lu}

\end{itemize}

\vspace*{1.5em}

Sincerely,\\[2em]
Mr. Gustavo Jeuken and\\
Prof. Lukas K\"all, \\
Science~for~Life~Laboratory, School of Biotechnology,\\
Royal Institute of Technology - KTH, 17165 Solna, Sweden\\
\url{lukas.kall@scilifelab.se}\\
Tel: +46 8 52481196, Fax: +46~8~52481425

\end{letter}
\end{document}
